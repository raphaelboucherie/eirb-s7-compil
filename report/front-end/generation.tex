\newpage
\section{Génération de RI}
\subsection{Structure de la représentation intermediaire}

Voici un tableau exposant des morceaux de code du langage d'entrée et leur correspondances en code 2 adresses : \\

\begin{tabular}{|l|l|l|}
  \hline
  \textbf{Type instruction} & \textbf{Langage d'entrée} & \textbf{Code 2 adresses} \\
  \hline
  \textbf{Affectation} & int i; & int i; \tabularnewline
    & i=0; & i=0; \tabularnewline
  \hline
  \textbf{Expression arithmetique} & c = a + b * f; & int tmp1; \tabularnewline
   & & tmp1 = b; \tabularnewline
   & & tmp1 *= f; \tabularnewline
   & & tmp1 += a; \tabularnewline
   & & c = tmp1; \tabularnewline
  \hline
  \textbf{Structure conditionnelle} & \raggedleft $if(i==10)\{$ & $if(i==10)\{$ \tabularnewline
  & $\ \ \ printf("Toto");$ & \raggedleft $\ \ \ printf("Toto");$ \tabularnewline
  & $\} else \{$ & \raggedleft $\}$ \tabularnewline
  & $\ \ \ printf("Titi");$ & \raggedleft $if(i!=10)\{$ \tabularnewline
  & $\} $ & \raggedleft $\ \ \ printf("Titi");$ \tabularnewline
  & & \raggedleft $\}$ \tabularnewline
  \hline
  \textbf{Boucle itérative} & \raggedleft $for(i=0; i<10; i++)\{$ & $i = 0;$ \tabularnewline
  \textbf{(for ou while)} & $\ \ \ d += 3;$ & \raggedleft $.for\_tmp\ :$ \tabularnewline
  & $\}$ & \raggedleft $\ \ if(i<10)\{$ \tabularnewline
  & & \raggedleft $\ \ \ \ \ d += 3;$ \tabularnewline
  & & \raggedleft $\ \ \ \ \ i++;$ \tabularnewline
  & & \raggedleft $\ \ \ \ \ goto\ .for\_tmp;$  \tabularnewline
  & & \raggedleft $\ \ \}$ \tabularnewline
  \hline
\end{tabular}

~~\\
La génération de la représentation intermediaire se réalise en parcourant l'arbre syntaxique comme suit :
[TODO]

~~\\
La génération du code 2 adresses passe par la génération de variables temporaires (ajoutées dans la table des symboles) et déclarées en début de bloc. 
[TODO]




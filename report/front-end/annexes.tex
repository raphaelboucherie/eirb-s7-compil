\newpage
\section{Annexes}

\subsection{Algorithme de verification des types}

\begin{algorithm}
\label{algo_verif_types}
\caption{check\_type(tree, symTable) : Algorithme de verification des types}
\label{algo_verif_types}
\begin{algorithmic}
\REQUIRE $tree\ (syntaxic\_tree)$, $symTable (symbols\_table)$
\ENSURE $type\ (type\ de\ retour\ de\ l'expression\ ou\ UNDEF\ si\ l'expression\ est\ invalide)$
\COMMENT{$On\ evalue\ d'abord\ les\ expressions\ situees\ au\ plus\ profond\ de\ l'arbre$}
\IF {$tree.left.type = TYPE\_UNDEF$\\ \textbf{or} $tree.right.type = TYPE\_UNDEF$}
\RETURN $TYPE\_UNDEF$
\ENDIF	
\IF {$tree\_length(tree.left) < tree\_length(tree.right)$}
\IF {$tree.right \neq NULL$}
\STATE $type\_right = check\_type(tree.right,\ symtable)$
\ENDIF
\IF {$tree.left \neq NULL$}
\STATE $type\_left = check\_type(tree.left,\ symtable)$
\ENDIF
\ELSE
\IF {$tree.left \neq NULL$}
\STATE $type\_left = check\_type(tree.left,\ symtable)$
\ENDIF
\IF {$tree.right \neq NULL$}
\STATE $type\_right = check\_type(tree.right,\ symtable)$
\ENDIF
\ENDIF
\STATE $ $
\COMMENT{$On\ controle\ la\ compatibilite\ des\ operandes$}
\IF {$is\_operator(tree.content)$} 
\IF {$tree.content = OP\_PLUS$} 
\IF {$tree.left.type = TYPE\_FLOAT$\\ \textbf{and} $tree.right.type = TYPE\_FLOAT$} 
\RETURN $TYPE\_FLOAT$
\ENDIF
\COMMENT{$...$}
\ELSE
\RETURN $TYPE\_UNDEF$
\ENDIF
\IF {$tree.content = OP\_MINUS$} 
\IF {$tree.left.type = TYPE\_INT$\\ \textbf{and} $tree.right.type = TYPE\_INT$} 
\RETURN $TYPE\_INT$
\ENDIF
\COMMENT{$...$}
\ELSE
\RETURN $TYPE\_UNDEF$
\ENDIF
\COMMENT{$...$}
\STATE $ $
\COMMENT{$On\ controle\ l'existence\ de\ l'operande\ dans\ la\ table\ des\ symboles$}
\ELSIF {$is\_operand(tree.content)$}
\IF {$is\_in\_table(tree.content, symTable)$} 
\RETURN $tree.content.type$
\ELSE
\PRINT ERROR 
\ENDIF
\ENDIF
\end{algorithmic}
\end{algorithm}
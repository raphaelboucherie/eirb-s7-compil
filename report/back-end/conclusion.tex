\section{Conclusion}

Le backend offre la possibilité de compiler une version hybride du C, certaines fonctionalitées comme les structures ou les switch n'étant pas implémentées. Cependant il offre des opérations sur les vecteurs que le C ne fournit pas et qui seront probablement plus rapide qu'une version écrite en C. Ce code présente cependant des limites comme la non gestion des types pointeurs qui sont un mechanisme fondamental du langage C, il à également été produit avec peu d'analyse préalable ce qui l'a rendus parfois peu lisible pour les personnes n'ayant pas travaillé dessus. Il reste néanmoins fonctionnel en grande partie et à principalement souffert de manque de documentation sur certains point de l'asm AT\& T. 
\section{Conclusion}

\vspace{0.5cm}
Le backend offre la possibilité de compiler une version hybride du C, certaines fonctionnalités comme les structures ou les switch n'étant pas implémentées. Cependant il offre des opérations sur les vecteurs que le C ne fournit pas et qui seront probablement plus rapide qu'une version écrite en C. 

\vspace{0.5cm}
Ce code présente cependant des limites comme la non gestion des types pointeurs qui sont un mécanisme fondamental du langage C. Il a également été produit avec peu d'analyse préalable ce qui l'a rendu parfois peu lisible pour les personnes n'ayant pas travaillé dessus. Il reste néanmoins fonctionnel en grande partie et a principalement souffert du manque de documentation sur certains points de l'asm AT\& T. 